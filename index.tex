% Options for packages loaded elsewhere
% Options for packages loaded elsewhere
\PassOptionsToPackage{unicode}{hyperref}
\PassOptionsToPackage{hyphens}{url}
\PassOptionsToPackage{dvipsnames,svgnames,x11names}{xcolor}
%
\documentclass[
  letterpaper,
  DIV=11,
  numbers=noendperiod]{scrreprt}
\usepackage{xcolor}
\usepackage{amsmath,amssymb}
\setcounter{secnumdepth}{5}
\usepackage{iftex}
\ifPDFTeX
  \usepackage[T1]{fontenc}
  \usepackage[utf8]{inputenc}
  \usepackage{textcomp} % provide euro and other symbols
\else % if luatex or xetex
  \usepackage{unicode-math} % this also loads fontspec
  \defaultfontfeatures{Scale=MatchLowercase}
  \defaultfontfeatures[\rmfamily]{Ligatures=TeX,Scale=1}
\fi
\usepackage{lmodern}
\ifPDFTeX\else
  % xetex/luatex font selection
\fi
% Use upquote if available, for straight quotes in verbatim environments
\IfFileExists{upquote.sty}{\usepackage{upquote}}{}
\IfFileExists{microtype.sty}{% use microtype if available
  \usepackage[]{microtype}
  \UseMicrotypeSet[protrusion]{basicmath} % disable protrusion for tt fonts
}{}
\makeatletter
\@ifundefined{KOMAClassName}{% if non-KOMA class
  \IfFileExists{parskip.sty}{%
    \usepackage{parskip}
  }{% else
    \setlength{\parindent}{0pt}
    \setlength{\parskip}{6pt plus 2pt minus 1pt}}
}{% if KOMA class
  \KOMAoptions{parskip=half}}
\makeatother
% Make \paragraph and \subparagraph free-standing
\makeatletter
\ifx\paragraph\undefined\else
  \let\oldparagraph\paragraph
  \renewcommand{\paragraph}{
    \@ifstar
      \xxxParagraphStar
      \xxxParagraphNoStar
  }
  \newcommand{\xxxParagraphStar}[1]{\oldparagraph*{#1}\mbox{}}
  \newcommand{\xxxParagraphNoStar}[1]{\oldparagraph{#1}\mbox{}}
\fi
\ifx\subparagraph\undefined\else
  \let\oldsubparagraph\subparagraph
  \renewcommand{\subparagraph}{
    \@ifstar
      \xxxSubParagraphStar
      \xxxSubParagraphNoStar
  }
  \newcommand{\xxxSubParagraphStar}[1]{\oldsubparagraph*{#1}\mbox{}}
  \newcommand{\xxxSubParagraphNoStar}[1]{\oldsubparagraph{#1}\mbox{}}
\fi
\makeatother


\usepackage{longtable,booktabs,array}
\usepackage{calc} % for calculating minipage widths
% Correct order of tables after \paragraph or \subparagraph
\usepackage{etoolbox}
\makeatletter
\patchcmd\longtable{\par}{\if@noskipsec\mbox{}\fi\par}{}{}
\makeatother
% Allow footnotes in longtable head/foot
\IfFileExists{footnotehyper.sty}{\usepackage{footnotehyper}}{\usepackage{footnote}}
\makesavenoteenv{longtable}
\usepackage{graphicx}
\makeatletter
\newsavebox\pandoc@box
\newcommand*\pandocbounded[1]{% scales image to fit in text height/width
  \sbox\pandoc@box{#1}%
  \Gscale@div\@tempa{\textheight}{\dimexpr\ht\pandoc@box+\dp\pandoc@box\relax}%
  \Gscale@div\@tempb{\linewidth}{\wd\pandoc@box}%
  \ifdim\@tempb\p@<\@tempa\p@\let\@tempa\@tempb\fi% select the smaller of both
  \ifdim\@tempa\p@<\p@\scalebox{\@tempa}{\usebox\pandoc@box}%
  \else\usebox{\pandoc@box}%
  \fi%
}
% Set default figure placement to htbp
\def\fps@figure{htbp}
\makeatother





\setlength{\emergencystretch}{3em} % prevent overfull lines

\providecommand{\tightlist}{%
  \setlength{\itemsep}{0pt}\setlength{\parskip}{0pt}}



 


\KOMAoption{captions}{tableheading}
\makeatletter
\@ifpackageloaded{bookmark}{}{\usepackage{bookmark}}
\makeatother
\makeatletter
\@ifpackageloaded{caption}{}{\usepackage{caption}}
\AtBeginDocument{%
\ifdefined\contentsname
  \renewcommand*\contentsname{Table of contents}
\else
  \newcommand\contentsname{Table of contents}
\fi
\ifdefined\listfigurename
  \renewcommand*\listfigurename{List of Figures}
\else
  \newcommand\listfigurename{List of Figures}
\fi
\ifdefined\listtablename
  \renewcommand*\listtablename{List of Tables}
\else
  \newcommand\listtablename{List of Tables}
\fi
\ifdefined\figurename
  \renewcommand*\figurename{Figure}
\else
  \newcommand\figurename{Figure}
\fi
\ifdefined\tablename
  \renewcommand*\tablename{Table}
\else
  \newcommand\tablename{Table}
\fi
}
\@ifpackageloaded{float}{}{\usepackage{float}}
\floatstyle{ruled}
\@ifundefined{c@chapter}{\newfloat{codelisting}{h}{lop}}{\newfloat{codelisting}{h}{lop}[chapter]}
\floatname{codelisting}{Listing}
\newcommand*\listoflistings{\listof{codelisting}{List of Listings}}
\makeatother
\makeatletter
\makeatother
\makeatletter
\@ifpackageloaded{caption}{}{\usepackage{caption}}
\@ifpackageloaded{subcaption}{}{\usepackage{subcaption}}
\makeatother
\usepackage{bookmark}
\IfFileExists{xurl.sty}{\usepackage{xurl}}{} % add URL line breaks if available
\urlstyle{same}
\hypersetup{
  pdftitle={Portofolio M. Abizzar G.},
  pdfauthor={13523155 - M. Abizzar Gamadrian},
  colorlinks=true,
  linkcolor={blue},
  filecolor={Maroon},
  citecolor={Blue},
  urlcolor={Blue},
  pdfcreator={LaTeX via pandoc}}


\title{Portofolio M. Abizzar G.}
\usepackage{etoolbox}
\makeatletter
\providecommand{\subtitle}[1]{% add subtitle to \maketitle
  \apptocmd{\@title}{\par {\large #1 \par}}{}{}
}
\makeatother
\subtitle{Seorang mahasiswa naif yang berusaha menjadi yang terbaik}
\author{13523155 - M. Abizzar Gamadrian}
\date{2025-10-21}
\begin{document}
\maketitle

\renewcommand*\contentsname{Table of contents}
{
\hypersetup{linkcolor=}
\setcounter{tocdepth}{2}
\tableofcontents
}

\bookmarksetup{startatroot}

\chapter*{Selamat Datang}\label{selamat-datang}
\addcontentsline{toc}{chapter}{Selamat Datang}

\markboth{Selamat Datang}{Selamat Datang}

\begin{figure}[H]

{\centering \includegraphics[width=9.5\linewidth,height=\textheight,keepaspectratio]{images/profile.jpg}

}

\caption{Foto M. Abizzar Gamadrian}

\end{figure}%

Selamat datang di portofolio digital saya.

Saya \textbf{M. Abizzar Gamadrian (13523155)}, dan ini adalah tagline
saya: ``seorang mahasiswa naif yang berusaha menjadi yang terbaik.''

Website ini adalah kumpulan cerita, refleksi, dan analisis diri saya
untuk memenuhi tugas mata kuliah II2100 Komunikasi Interpersonal dan
Publik.

\subsection*{Mulai dari Mana?}\label{mulai-dari-mana}
\addcontentsline{toc}{subsection}{Mulai dari Mana?}

\begin{itemize}
\tightlist
\item
  Untuk perkenalan lengkap tentang siapa saya, silakan kunjungi:

  \begin{itemize}
  \tightlist
  \item
    \textbf{\href{./All_About_me/index.qmd}{UTS-1: All About Me}}
  \end{itemize}
\item
  Untuk cerita paling transformatif dalam hidup saya:

  \begin{itemize}
  \tightlist
  \item
    \textbf{\href{./My_Stories_for_You/index.qmd}{UTS-3: My Stories for
    You}}
  \end{itemize}
\item
  Untuk analisis mendalam tentang kepribadian (ENFP) dan kekuatan saya:

  \begin{itemize}
  \tightlist
  \item
    \textbf{\href{./My_Shapes/index.qmd}{UTS-4: My SHAPE}}
  \end{itemize}
\end{itemize}

Anda juga dapat menggunakan menu navigasi di samping untuk menjelajahi
bagian lainnya. Terima kasih telah berkunjung.

\bookmarksetup{startatroot}

\chapter{UTS-1: All About Me}\label{uts-1-all-about-me}

Selamat datang di perkenalan resmi saya. Untuk benar-benar mengenal saya
di luar nama dan NIM, ada beberapa hal yang perlu Anda ketahui.

\subsection{Fakta Unik: Tiga Keanehan
Saya}\label{fakta-unik-tiga-keanehan-saya}

Mari kita mulai dengan hal-hal yang tidak biasa---bagian yang
menargetkan rubrik ``Humor'' dan ``Orisinalitas''.

\begin{enumerate}
\def\labelenumi{\arabic{enumi}.}
\tightlist
\item
  \textbf{Saya Buta Warna Parsial:} Ini bukan halangan, tapi ini membuat
  saya melihat dunia sedikit berbeda (secara harfiah). Ini juga
  menjelaskan mengapa saya mungkin tidak akan pernah berdebat dengan
  Anda tentang warna \emph{teal} vs \emph{turquoise}.
\item
  \textbf{Saya Menggunakan Takdir untuk Keputusan Kecil:} Saya benci
  terjebak dalam pilihan sepele (mau makan apa, nonton apa). Jika saya
  benar-benar bingung, saya akan membuka \emph{spin-wheel} di internet,
  memasukkan pilihannya, dan membiarkan roda digital itu memutuskan
  takdir saya.
\item
  \textbf{Saya Pernah Hampir Tenggelam:} Waktu kecil, saya dengan
  percaya diri mencoba menyeberangi kolam yang dalam. Tentu saja, saya
  gagal. Untungnya, sepupu saya melihat dan menyelamatkan saya. Ini
  adalah pelajaran hidup yang brutal namun efektif tentang realitas.
\end{enumerate}

\subsection{Keseimbangan ENFP: Dunia di Waktu
Luang}\label{keseimbangan-enfp-dunia-di-waktu-luang}

Fakta-fakta acak di atas sebenarnya sangat mencerminkan kepribadian saya
sebagai seorang \textbf{ENFP-A (Juru Kampanye)}. Sebagai seorang ENFP,
saya adalah seorang ekstrovert yang berkembang melalui hubungan
(kekuatan VIA \#1 saya adalah \textbf{Cinta (Love)}), tetapi saya juga
seorang intuitif (N) yang mendambakan kedalaman makna.

Di waktu luang, dunia saya terbagi dua:

\begin{enumerate}
\def\labelenumi{\arabic{enumi}.}
\item
  \textbf{Dunia Epik (Pelarian Intuitif):} Saya sangat menikmati narasi
  yang kompleks dan epik. Saya adalah fans berat \emph{Attack on
  Titan}---bagi saya, itu bukan sekadar anime, tapi sebuah mahakarya
  tentang filosofi dan kondisi manusia. Saya juga sangat menyukai
  penulisan karakter yang brilian di \emph{Better Call Saul}.
\item
  \textbf{Dunia Fisik (Keseimbangan Energi):} Saya bukan tipe orang yang
  suka lari maraton (aerobik). Saya lebih suka olahraga anaerobik,
  secara spesifik \textbf{Calisthenics}. Ada kepuasan tersendiri dalam
  melatih kekuatan tubuh secara terkontrol. Dan ya, jika saya sedang
  penat atau stres karena kuliah, saya suka bernyanyi-nyanyi. Mungkin
  tidak merdu, tapi efektif untuk meredakan stres.
\end{enumerate}

\subsection{Wawasan Saya: Prinsip
Anti-Tenggelam}\label{wawasan-saya-prinsip-anti-tenggelam}

Jika ada satu benang merah yang mengikat semua ini, itu adalah kutipan
dari Mary Pickford yang saya pegang teguh:

\begin{quote}
``This thing that we call failure is not the falling down, but the
staying down.''
\end{quote}

Bagi saya, ini bukan sekadar kata-kata. Ini adalah prinsip hidup. Ini
adalah pengingat bahwa gagal itu tidak apa-apa. Lelah itu wajar. Hampir
tenggelam di kolam renang itu memalukan, tapi itu bagian dari proses.

Selama saya tidak memilih untuk ``tetap di bawah''---baik itu di kolam
renang atau saat menghadapi kegagalan UTBK (yang bisa Anda baca di
\href{../My_Stories_for_You/index.qmd}{My Stories for You})---maka saya
belum benar-benar gagal.

\bookmarksetup{startatroot}

\chapter{UTS-2: Sebuah Lagu Untukmu}\label{uts-2-sebuah-lagu-untukmu}

Pesan ini saya tujukan kepada orang-orang terdekat saya---keluarga dan
sahabat---yang selalu ada bersama saya, bahkan ketika saya melakukan
kesalahan dan belum menjadi yang terbaik.

Saya memilih lagu ``If We Have Each Other'' dari Alec Benjamin.

\subsection{Kenapa Lagu Ini?}\label{kenapa-lagu-ini}

Alasan saya memilih lagu ini sederhana: pesannya sangat kuat. Lagu ini
bercerita bahwa tidak peduli seberapa sulit, kacau, atau tidak
sempurnanya dunia ini, selama kita memiliki satu sama lain, kita akan
baik-baik saja.

Ini adalah janji sekaligus pengingat untuk mereka:

\begin{quote}
``You should know I'll be there for you.''
\end{quote}

Jika mereka sedang kesusahan, mereka harus tahu bahwa saya akan ada
untuk mereka, sama seperti mereka selalu ada untuk saya.

Lirik lagu by Musixmatch:

She was 19 with a baby on the way\\
On the East-side of the city, she was working every day\\
Cleaning dishes in the evening, she could barely stay awake\\
She was clinging to the feeling that her luck was gonna change\\
And, 'cross town she would take the bus at night\\
To a one-bedroom apartment, and when she'd turn on the light\\
She would sit down at the table, tell herself that it's alright\\
She was waiting on the day she hoped her baby would arrive\\
She'd never be alone\\
Have someone to hold\\
And when nights were cold\\
She'd say\\
The world's not perfect, but it's not that bad\\
If we got each other, and that's all we have\\
I will be your mother, and I'll hold your hand\\
You should know I'll be there for you\\
When the world's not perfect, when the world's not kind\\
If we have each other, then we'll both be fine\\
I will be your mother, and I'll hold your hand\\
You should know I'll be there for you\\
They were 90 and were living out their days\\
On the West-side of the city next to where they got engaged\\
They had pictures on the walls of all the memories that they'd made\\
And though life was never easy, they were thankful that they stayed\\
With each other, and though some times were hard\\
Even when she made him angry, he would never break her heart\\
No, they didn't have the money to afford a fancy car\\
But they never had to travel 'cause they'd never be apart\\
Even at the end\\
Their love was stronger than\\
The day that they first met\\
They'd say\\
The world's not perfect, but it's not that bad\\
If we got each other, and that's all we have\\
I will be your lover, and I'll hold your hand\\
You should know I'll be there for you\\
When the world's not perfect, when the world's not kind\\
If we have each other, then we'll both be fine\\
I will be your lover, and I'll hold your hand\\
You should know I'll be there for you\\
You should know I'll be there for you\\
I'm 23, and my folks are getting old\\
I know they don't have forever, and I'm scared to be alone\\
So I'm thankful for my sister, even though sometimes we fight\\
When high school wasn't easy, she's the reason I survived

\textbar I know she'd never leave me and I hate to see her cry
\textbar So I wrote this verse to tell her that I'm always by her side
\textbar I wrote this verse to tell her that I'm always by her side
\textbar I wrote this verse to tell her that \textbar The world's not
perfect, but it's not that bad \textbar If we got each other, and that's
all we have \textbar I will be your brother, and I'll hold your hand
\textbar You should know I'll be there for you \textbar When the world's
not perfect, when the world's not kind \textbar{} If we have each other,
then we'll both be fine \textbar{} I will be your brother, and I'll hold
your hand \textbar{} You should know I'll be there for you \textbar{}
You should know I'll be there for you

\bookmarksetup{startatroot}

\chapter{UTS-3: Kisah Jatuh di Garis Start (dan Menolak untuk Tetap di
Bawah)}\label{uts-3-kisah-jatuh-di-garis-start-dan-menolak-untuk-tetap-di-bawah}

Di halaman utama
\hyperref[uts-3-kisah-jatuh-di-garis-start-dan-menolak-untuk-tetap-di-bawah]{portofolio
ini}, saya membagikan kutipan favorit saya dari Mary Pickford:

\begin{quote}
``This thing that we call failure is not the falling down, but the
staying down.''
\end{quote}

Kutipan itu bukan sekadar teori atau kata-kata indah yang saya temukan
di internet. Bagi saya, itu adalah pengalaman pribadi. Ini adalah cerita
tentang kegagalan terbesar saya.

\subsection{Bagian 1: Harapan dan
Beban}\label{bagian-1-harapan-dan-beban}

Saya ingat dengan jelas perasaan saya menjelang Ujian Tulis Berbasis
Komputer (UTBK). Ini bukan sekadar ujian masuk universitas biasa. Bagi
saya, ini adalah momen pembuktian.

Orang tua saya telah berkorban begitu besar. Mereka menyekolahkan saya
jauh-jauh ke Palembang, ke salah satu SMA terbaik di provinsi Sumatera
Selatan. Mereka membiayai kursus tambahan, memberikan semua yang saya
butuhkan, dan mendukung penuh pembelajaran saya. Saya membawa harapan
dan investasi mereka di pundak saya. Lulus UTBK di pilihan pertama atau
kedua adalah satu-satunya skenario yang ada di kepala saya.

\subsection{Bagian 2: Layar Merah
Kegagalan}\label{bagian-2-layar-merah-kegagalan}

Lalu, hari pengumuman itu tiba.

Saya membuka \emph{website} pengumuman. Jantung saya berdebar kencang.
Saya memasukkan nomor peserta saya, menekan `Enter', dan menahan napas.

Yang muncul adalah warna merah. Tulisan ``Anda dinyatakan TIDAK
LULUS\ldots{}''

Saya tidak lulus. Pilihan satu dan pilihan dua, keduanya menolak saya.

Dunia saya seakan berhenti berputar. Perasaan pertama yang muncul
bukanlah kesedihan, tapi \textbf{malu}. Saya merasa telah gagal total
sebagai seorang anak. Saya telah mengecewakan orang-orang yang paling
saya cintai; mereka yang telah memberikan segalanya untuk saya.

Saya duduk terdiam untuk waktu yang lama. Hal tersulit bukanlah melihat
layar merah itu, tapi memikirkan bagaimana cara mengabari orang tua
saya. Saya tidak berani. Saya malu membayangkan nada kecewa dalam suara
mereka.

\subsection{Bagian 3: Titik Balik (Kekuatan dari
``Cinta'')}\label{bagian-3-titik-balik-kekuatan-dari-cinta}

Ketika saya akhirnya mengumpulkan keberanian untuk menelepon ke rumah,
saya sudah bersiap untuk dimarahi, atau setidaknya, mendengar kekecewaan
yang mendalam.

Apa yang saya terima justru sebaliknya.

Orang tua saya sangat suportif. Tentu, mereka kaget, tapi tidak ada
amarah. Kalimat pertama yang mereka ucapkan adalah, ``Tidak apa-apa. Ini
bukan akhir.'' Mereka tidak melihat saya sebagai kegagalan. Mereka
melihat saya sebagai anak mereka yang sedang ``jatuh''.

Dukungan tanpa syarat itulah yang menjadi titik balik saya. Di situlah
saya memutuskan untuk tidak ``tetap di bawah''.

\subsection{Bagian 4: Maraton 6 Ujian}\label{bagian-4-maraton-6-ujian}

Saya tidak membiarkan kesedihan berlarut-larut. Kegagalan UTBK adalah
fakta, tapi itu bukan takdir.

Orang tua saya mendorong saya untuk mencari jalan lain. Akhirnya, saya
mendaftar dan mempersiapkan diri untuk \textbf{enam} ujian mandiri di
berbagai kampus. Bulan berikutnya adalah maraton belajar yang
sesungguhnya. Saya percaya Tuhan memberi saya jalan yang penuh tantangan
ini karena saya bisa melewatinya. Saya kembali belajar dengan giat, kali
ini didorong bukan oleh rasa takut gagal, tapi oleh keinginan untuk
membuktikan bahwa dukungan orang tua saya tidak sia-sia.

Satu per satu, pengumuman ujian mandiri mulai keluar.

\subsection{Bagian 5: Resolusi dan
Pelajaran}\label{bagian-5-resolusi-dan-pelajaran}

Hasilnya: Saya lulus di \textbf{5 dari 6} kampus yang saya daftarkan.

Dan ironisnya, salah satu kampus yang menerima saya melalui jalur
mandiri adalah universitas yang sama yang menolak saya mentah-mentah di
UTBK. Saya berhasil mendapatkannya kembali, bukan karena keberuntungan,
tapi karena saya tidak menyerah untuk mencoba lagi.

Pelajaran terbesar yang saya dapatkan adalah validasi mutlak dari
kutipan itu.

Gagal UTBK adalah momen `jatuh' (falling down). Itu menyakitkan,
memalukan, dan brutal. Tapi itu hanya sebuah peristiwa.

Memutuskan untuk berhenti mencoba, menyalahkan takdir, dan menyerah pada
rasa malu---itulah yang disebut `tetap di bawah' (staying down). Dan
saya memilih untuk tidak melakukannya.

\bookmarksetup{startatroot}

\chapter{UTS-4: Membedah SHAPE Saya}\label{uts-4-membedah-shape-saya}

Di bagian ini, saya melakukan refleksi diri berdasarkan kerangka kerja
SHAPE (Strengths, Heart, Aptitudes, Personality, Experiences). Ini
adalah upaya untuk memahami ``manual'' diri saya sendiri, dengan segala
kelebihan dan kontradiksi menarik di dalamnya.

\subsection{S - Strengths (Kekuatan)}\label{s---strengths-kekuatan}

Saya mengambil asesmen VIA Character Strengths. Hasilnya sangat selaras
dengan apa yang saya rasakan.

\textbf{5 Kekuatan Teratas Saya:} \textbar{} 1. \textbf{Cinta (Love):}
Menghargai hubungan dekat dengan orang lain. \textbar{} 2.
\textbf{Perspektif (Perspective):} Mampu memberikan nasihat bijak.
\textbar{} 3. \textbf{Penilaian (Judgment):} Berpikir kritis dan
menimbang dari semua sisi. \textbar{} 4. \textbf{Kejujuran (Honesty):}
Bertindak tulus dan apa adanya. \textbar{} 5. \textbf{Kebaikan
(Kindness):} Senang membantu dan berbuat baik.

\textbf{Refleksi ``Keautentikan'':} Saya sangat setuju dengan 5 kekuatan
teratas ini. Kekuatan \textbf{Cinta (Love)}, \textbf{Kebaikan
(Kindness)}, dan \textbf{Kejujuran (Honesty)} adalah fondasi saya. Saya
selalu berusaha jujur, yang terbukti sejak dulu saat saya memilih untuk
juara kelas 100\% tanpa mencontek. Mencapai sesuatu dengan jujur terasa
jauh lebih memuaskan.

Kekuatan \textbf{Perspektif} dan \textbf{Penilaian} juga sangat ``saya
banget''. Ini terhubung langsung dengan bakat saya sebagai pendengar
yang baik; saya tidak hanya mendengar, tetapi juga mencoba memberikan
perspektif baru atas masalah yang diceritakan teman saya.

\subsection{H - Heart (Hati: Nilai \&
Gairah)}\label{h---heart-hati-nilai-gairah}

Hasil tes Personal Values saya menunjukkan 5 nilai teratas yang memandu
hidup saya: \textbar{} 1. \textbf{Certainty (Kepastian):} Saya menyukai
stabilitas, keteraturan, dan prediktabilitas. Saya merasa nyaman dan
aman ketika segala sesuatunya jelas. \textbar{} 2. \textbf{Financial
Stability (Stabilitas Finansial):} Bagi saya, uang adalah instrumen
untuk memenuhi kebutuhan dan mimpi. Saya sangat rapi dalam mengelola
keuangan bulanan untuk mencapai ini. \textbar{} 3. \textbf{Success
(Sukses):} Mencapai hasil yang diinginkan adalah konfirmasi bahwa
keputusan saya benar, yang membangun kepuasan dan harga diri. \textbar{}
4. \textbf{Family (Keluarga):} Tempat saya menemukan cinta, keamanan,
dan motivasi. \textbar{} 5. \textbf{Friendship (Persahabatan):} Saya
menghargai ikatan yang kuat, rasa saling percaya, dan kebersamaan dengan
orang-orang baik.

Nilai \textbf{Family} dan \textbf{Friendship} adalah manifestasi nyata
dari kekuatan VIA \#1 saya, yaitu \textbf{Cinta (Love)}.

\subsection{A - Aptitudes (Bakat \&
Keterampilan)}\label{a---aptitudes-bakat-keterampilan}

Ini adalah hal-hal yang saya kuasai, baik secara teknis maupun
interpersonal:

\textbf{Keterampilan Teknis \& Analitis:} \textbar{} * Bisa coding
\textbar{} * Bisa mengedit video (jika ada bahan dan ide) \textbar{} *
Pandai mengelola keuangan pribadi/bulanan dengan rapi \textbar{} * Jago
mengingat rute jalan dan arah (tidak mudah tersesat)

\textbf{Keterampilan Interpersonal (ENFP-style):} \textbar{} * Mudah
mengobrol 1-on-1 dengan orang baru \textbar{} * Pandai mendengarkan
curhat orang lain (pendengar yang baik) \textbar{} * Bisa memberikan
perspektif baru saat orang lain bercerita \textbar{} * Jago main game
(seringkali membutuhkan strategi dan kerja sama tim) \textbar{} * Bisa
Calisthenics dasar

\subsection{P - Personality
(Kepribadian)}\label{p---personality-kepribadian}

Hasil tes MBTI saya adalah \textbf{ENFP-A (Juru Kampanye)}.

Ini sangat menjelaskan mengapa kekuatan utama saya adalah \textbf{Cinta
(Love)} dan \textbf{Kebaikan (Kindness)}. Sebagai seorang Ekstrovert (E)
dan Perasa (F), saya mendapatkan energi dari interaksi sosial yang
bermakna dan sangat peduli pada hubungan. Ini juga menjelaskan mengapa
saya mudah mengobrol 1-on-1 dan menjadi pendengar yang baik.

\textbf{Refleksi ``Keautentikan'': Paradoks Diri Saya} Hal yang paling
mengejutkan sekaligus paling ``saya banget'' adalah kontradiksi antara
\textbf{Nilai (Heart)} dan \textbf{Kepribadian (Personality)} saya.

\begin{itemize}
\tightlist
\item
  Nilai \#1 saya adalah \textbf{Certainty (Kepastian)}. Saya suka hal
  yang jelas dan terencana.
\item
  Tapi, hasil MBTI saya menunjukkan saya 63\% \textbf{Improvising
  (Prospecting/P)}, yang artinya saya lebih dominan improvisasi daripada
  perencanaan kaku.
\end{itemize}

Refleksi saya adalah: Saya adalah seorang \textbf{``Improviser yang
Mencari Kepastian''}. Saya tidak suka rencana yang kaku, tapi saya
\emph{butuh} tujuan yang pasti. Saya mungkin fleksibel dan jago
improvisasi \emph{dalam perjalanan}, tapi saya harus tahu \emph{ke mana}
saya akan pergi (Finansial Stabil, Sukses, dll). Ini adalah inti dari
diri saya: Fleksibel dalam cara, namun pasti dalam tujuan.

\subsection{E - Experiences
(Pengalaman)}\label{e---experiences-pengalaman}

Dua pengalaman hidup ini sangat membentuk saya: \textbar{} 1.
\textbf{Juara Kelas dengan Kejujuran:} Saat sekolah, saya pernah juara
kelas 100\% jujur tanpa mencontek. Pengalaman itu menanamkan nilai bahwa
\textbf{Kejujuran} (kekuatan VIA saya) jauh lebih memuaskan daripada
pencapaian instan lewat kebohongan. \textbar{} 2. \textbf{Orang Datang
dan Pergi:} Saya dulu sering punya teman yang sangat dekat, namun
seiring berjalannya waktu, hubungan itu memudar bukan karena konflik,
tapi karena keadaan dan kesibukan. Ini mengajarkan saya pelajaran
berharga: \emph{``It's okay that people come and go.''} Ini membuat saya
semakin menghargai \textbf{Keluarga} dan \textbf{Persahabatan} (nilai
saya) yang bertahan \emph{saat ini}.

\begin{center}\rule{0.5\linewidth}{0.5pt}\end{center}

\subsection{Piagam Diri (Self Charter)
Saya}\label{piagam-diri-self-charter-saya}

Berdasarkan semua analisis SHAPE ini, inilah Piagam Diri saya:

\begin{quote}
``Saya adalah \textbf{Juru Kampanye (ENFP)} yang beroperasi dengan inti
\textbf{Cinta (Love)} dan \textbf{Kejujuran (Honesty)}.

Kekuatan saya terletak pada kemampuan untuk \textbf{terhubung
(Kindness)} dengan orang lain dan memberikan \textbf{Perspektif
(Perspective)} yang jernih.

Saya adalah paradoks yang hidup: seorang \textbf{improviser (P)} yang
mendambakan \textbf{kepastian (Certainty)}. Saya fleksibel dalam metode,
tapi teguh pada nilai.

Saya dibentuk oleh \textbf{kegagalan (UTS-3)} dan \textbf{persahabatan},
yang mengajarkan saya bahwa jatuh itu wajar, dan bahwa orang boleh
datang dan pergi.

Tujuan saya adalah meraih \textbf{Sukses} dan \textbf{Stabilitas
Finansial} sambil tetap menjadi pendengar yang baik bagi
\textbf{Keluarga} dan \textbf{Sahabat} saya.''
\end{quote}

\bookmarksetup{startatroot}

\chapter{UTS-5 My Personal Reviews}\label{uts-5-my-personal-reviews}

Berikut adalah hasil self-assessment portofolio saya. Metode ini
mengikuti panduan tugas, di mana saya (sebagai penilai) menganalisis
karya saya sendiri menggunakan rubrik yang telah disediakan.

\bookmarksetup{startatroot}

\chapter{Hasil Self-Assessment UTS (URL:
https://abizzarg.github.io/all-about-me/)}\label{hasil-self-assessment-uts-url-httpsabizzarg.github.ioall-about-me}

\section{Identifikasi}\label{identifikasi}

\begin{itemize}
\tightlist
\item
  \textbf{Nama \& NIM penulis:} M. Abizzar Gamadrian -- 13523155
  (tertera di halaman depan portofolio).
\item
  \textbf{Penilai:} Self-assessment (M. Abizzar Gamadrian)
\item
  \textbf{Catatan cakupan:} Halaman beranda (\texttt{index.qmd}) memuat
  rangkuman ``About Me'' (UTS-1) dan ``Prinsip Saya'' (terhubung ke
  UTS-3). Navigasi di \emph{sidebar} lengkap ke semua chapter (UTS-1 s/d
  UTS-5).
\end{itemize}

\section{Tinjauan Umum}\label{tinjauan-umum}

\begin{itemize}
\tightlist
\item
  \textbf{UTS-1 (All About Me)} hadir di chapter
  \texttt{All\_About\_me/index.qmd} dan \texttt{index.qmd}. Konten
  sangat personal, memperkenalkan fakta unik (buta warna, spin-wheel),
  hobi (AoT, calisthenics), dan prinsip hidup (``staying down'').
\item
  \textbf{UTS-2 (My Songs for You)} memuat dedikasi yang jelas, video
  YouTube yang di-embed, narasi personal di balik pemilihan lagu, serta
  lirik lengkap. Konten sudah 100\% lengkap.
\item
  \textbf{UTS-3 (My Stories for You)} berisi satu narasi kuat tentang
  pengalaman kegagalan UTBK, dengan struktur cerita (konflik, resolusi)
  yang jelas.
\item
  \textbf{UTS-4 (My SHAPE)} berisi analisis komprehensif (S-H-A-P-E),
  menghubungkan data tes (VIA, ENFP, Values) dengan refleksi otentik
  (paradoks ENFP vs Certainty) dan ditutup Piagam Diri.
\item
  \textbf{UTS-5 (My Personal Reviews)} (halaman ini) berisi
  \emph{self-assessment} sesuai format. Bagian \emph{Peer-Assessment}
  akan ditambahkan kemudian.
\end{itemize}

\begin{center}\rule{0.5\linewidth}{0.5pt}\end{center}

\section{Tinjauan Spesifik + Skor
(1--5)}\label{tinjauan-spesifik-skor-15}

\subsection{UTS-1 --- All About Me}\label{uts-1-all-about-me-1}

\textbf{Skor per kriteria:} Orisinalitas \textbf{5}, Keterlibatan
\textbf{5}, Humor \textbf{4}, Wawasan/Insight \textbf{5} → \textbf{Total
19/20 (95\%)}. \textbf{Alasan singkat:} Konten sangat otentik dan tidak
kaku. Penggunaan fakta unik (spin-wheel) memenuhi kriteria ``Humor''.
Penggunaan kutipan ``staying down'' sebagai benang merah portofolio
menunjukkan ``Wawasan'' yang kuat. \textbf{Saran perbaikan:} Konten
UTS-1 sedikit terbagi antara \texttt{index.qmd} (Halaman Utama) dan
\texttt{All\_About\_me/index.qmd}. Pertimbangkan untuk menyatukan semua
teks perkenalan diri di dalam chapter \texttt{All\_About\_me/index.qmd}
agar lebih runut.

\subsection{UTS-2 --- My Songs for You}\label{uts-2-my-songs-for-you}

\textbf{Skor per kriteria:} Orisinalitas \textbf{4}, Keterlibatan
\textbf{5}, Humor \textbf{1}, Inspirasi \textbf{5} → \textbf{Total 15/20
(75\%)}. \textbf{Alasan singkat:} ``Keterlibatan'' dan ``Inspirasi''
sangat tinggi karena adanya narasi personal ``kenapa lagu ini''.
Penggunaan \emph{embed} video lebih baik daripada link mati. Skor
``Orisinalitas'' 4 (bukan 5) karena menggunakan karya orang lain (bukan
ciptaan sendiri). Skor ``Humor'' 1 karena pesan lagu ini memang tidak
ditujukan untuk lucu. Konten sekarang sudah lengkap dengan transkrip
lirik, membuat pesan tersampaikan dengan utuh. \textbf{Saran perbaikan:}
Tidak ada. Konten sudah lengkap.

\subsection{UTS-3 --- My Stories for
You}\label{uts-3-my-stories-for-you}

\textbf{Skor per kriteria:} Orisinalitas \textbf{5}, Keterlibatan
\textbf{5}, Pengembangan Narasi \textbf{5}, Inspirasi \textbf{5} →
\textbf{Total 20/20 (100\%)}. \textbf{Alasan singkat:} Cerita kegagalan
UTBK sangat personal, jujur (``malu mengabari orang tua''), dan memenuhi
rubrik ``Pengembangan Narasi'' (Setup: gagal, Conflict: malu \&
berjuang, Resolution: lulus 5/6). Poin ``Inspirasi'' maksimal karena
terhubung langsung dengan tema utama portofolio (``staying down'').
\textbf{Saran perbaikan:} Sudah sangat kuat. Tidak ada perbaikan mayor.

\subsection{UTS-4 --- My SHAPE}\label{uts-4-my-shape}

\textbf{Skor per kriteria:} Orisinalitas \textbf{5}, Keterlibatan
\textbf{5}, Keautentikan \textbf{5}, Inspirasi \textbf{5} →
\textbf{Total 20/20 (100\%)}. \textbf{Alasan singkat:} Ini adalah bagian
terkuat. Skor ``Keautentikan'' maksimal karena tidak hanya melaporkan
hasil tes, tetapi menganalisis paradoks (ENFP vs Certainty).
``Orisinalitas'' tinggi karena menghubungkan semua data (VIA, Aptitudes,
Experiences) menjadi ``Piagam Diri'' yang koheren. \textbf{Saran
perbaikan:} Untuk meningkatkan ``Keterlibatan'' visual, pertimbangkan
membuat 1 tabel ringkas di bagian atas yang merangkum hasil S-H-A-P-E
sebelum masuk ke narasi reflektif.

\subsection{UTS-5 --- My Personal
Reviews}\label{uts-5-my-personal-reviews-1}

\textbf{Skor per kriteria:} Pemahaman Konsep \textbf{5}, Analisis Kritis
\textbf{4}, Argumentasi (Logos) \textbf{4}, Etos \& Empati \textbf{N/A},
Rekomendasi \textbf{4} → \textbf{Total (Self-Assess) 17/20 (85\%)}.
\textbf{Alasan singkat:} Halaman ini (saat ini) telah berhasil memenuhi
tugas \emph{Self-Assessment} dengan ``Pemahaman Konsep'' rubrik,
melakukan ``Analisis Kritis'' pada setiap UTS, dan memberikan
``Rekomendasi'' yang spesifik. \textbf{Saran perbaikan:} Tugas ini belum
selesai. Perlu segera dilengkapi dengan: 1. Melakukan
\emph{Peer-Assessment} (menilai 2-3 portofolio rekan) dan menambahkannya
di bawah bagian ini. 2. Mengisi file \texttt{Lembar\ Skor.xlsx}
(termasuk skor \emph{peer-assessment}) dan meng-upload-nya ke folder
ini, lalu membuat \emph{link} unduhan.

\begin{center}\rule{0.5\linewidth}{0.5pt}\end{center}

\section{Rekap Skor (ringkas)}\label{rekap-skor-ringkas}

\begin{itemize}
\tightlist
\item
  \textbf{UTS-1:} 19/20 → \textbf{95\%}
\item
  \textbf{UTS-2:} 15/20 → \textbf{75\%}
\item
  \textbf{UTS-3:} 20/20 → \textbf{100\%}
\item
  \textbf{UTS-4:} 20/20 → \textbf{100\%}
\item
  \textbf{UTS-5:} (Baru Self-Assess) \textbf{17/20 (85\%)}
\end{itemize}

\emph{(Skor ini adalah hasil self-assessment dan akan dilengkapi dengan
Peer-Assessment)}

\section{Langkah Perbaikan Cepat
(Prioritas)}\label{langkah-perbaikan-cepat-prioritas}

\begin{enumerate}
\def\labelenumi{\arabic{enumi}.}
\tightlist
\item
  \textbf{Lengkapi UTS-5:} Ini prioritas utama. Lakukan
  \emph{Peer-Assessment} dan upload \texttt{Lembar\ Skor.xlsx}.
\item
  \textbf{Perbaiki UTS-1:} Satukan konten perkenalan diri ke dalam satu
  chapter \texttt{All\_About\_me/index.qmd}.
\end{enumerate}

\bookmarksetup{startatroot}

\chapter{UAS-1 My Concepts}\label{uas-1-my-concepts}

Mau hidup epik ? \href{lifestory.pdf}{Write your Life Story}

Apa itu berkonsep?

\url{https://youtu.be/QVfUlVBO80U?si=yM6q_rwV9rcDBbu7}

\bookmarksetup{startatroot}

\chapter{UAS-3 My Opinions}\label{uas-3-my-opinions}

SApa itu beropini? \href{BM\%20Opini.mp4}{Opini Berpengaruh}

Bagiamana menjaadi menarik? \href{./Interesting.mp4}{Menjadi Menarik}

\bookmarksetup{startatroot}

\chapter{UAS-3 My Innovations}\label{uas-3-my-innovations}

\bookmarksetup{startatroot}

\chapter{UAS-4 My Knowledge}\label{uas-4-my-knowledge}

Cara saya mengkomunikasikan sebuah pengatahuan sebagai petunjuk bagi
orang lain 1) saya tulis
\href{Rekomendasi\%20Presentasi\%20Efektif(Contoh\%20Makalah).pdf}{makalah
sebagai bahan utama} 2) lalu saya buat
\href{Contoh\%20Transkrip\%20Presentasi.pdf}{transkrip ucapan lisan} 3)
kemudian saya kembangkan
\href{Rekomendasi\%20Presentasi\%20(Contoh\%20Slides).pdf}{slide
pendukung trnsskrip} 4) lalu saya memproduksivideo audio visual
\url{https://youtu.be/ZbghfMvnPZc} \url{https://youtu.be/ZbghfMvnPZc}

\bookmarksetup{startatroot}

\chapter{UAS-5 My Professional
Reviews}\label{uas-5-my-professional-reviews}

Untuk melAkukan review, seperti pada
\href{../My_Personal_Reviews/Doc.5.Mengevaluasi-Esai-Berdasarkan-Rubrik.pdf}{pendekatan
AI}, kita membutuhkan rubrik

\bookmarksetup{startatroot}

\chapter{Summary}\label{summary}

In summary, this book has no content whatsoever.

\bookmarksetup{startatroot}

\chapter*{References}\label{references}
\addcontentsline{toc}{chapter}{References}

\markboth{References}{References}

\phantomsection\label{refs}




\end{document}
